\documentclass{ctexart}
\usepackage[margin=1in]{geometry}
\usepackage{listings}
\usepackage{xcolor}
\usepackage[colorlinks]{hyperref}

\ctexset{
  section/format = \Large\bfseries\raggedright
}

\lstset{
  backgroundcolor = \color{lightgray!30},
  keywordstyle = \color{blue},
  stringstyle = \color{orange},
  basicstyle = {\small\ttfamily},
  commentstyle = {\color{brown}\itshape},
  breaklines = true,
  tabsize = 4,
  gobble = 2,
  numbers = left,
  numberstyle = \tiny,
  emph={
    copy,call,IF,GOTO
  },
  emphstyle={\color{blue}\small\ttfamily}%
}

\title{简单记录一下我配置 MinGW 64 的过程}
\author{啸行\thanks{\url{ranwang.osbert@outlook.com}}}

\begin{document}
  
\maketitle

\begin{abstract}
  简单记录一下自己电脑上配置 MinGW 64 的过程,省的以后忘记。
\end{abstract}

\section{下载 MinGW 64}

首先下载 \href{https://sourceforge.net/projects/mingw-w64/files/mingw-w64/}{MinGW 64},选择 \texttt{x86\_64-posix-seh}。解压缩后,记录位置以备后续环境配置时使用。

\section{编写 makefile 文件}

我使用的 \texttt{makefile} 文件是通用的,默认生成 \texttt{main.exe}

\begin{lstlisting}[language=make]
  # 'wildcard' means all the C file in this folder
  CFILE = $(wildcard *.c)
  # 'patsubst' means turn all the C file in CFILE to the O files
  OFILE = $(patsubst %.c,%.o,$(CFILE))
  CC = gcc -std=c99
  # default FILENAME is main
  FILENAME ?= main
  
  # $@ -- target file; $^ -- all corresponding files; $< -- first corresponding files
  $(FILENAME): $(OFILE)
  	$(CC) $^ -o $@
  $(OFILE):%.o:%.c
  	$(CC) -c $< -o $@
   
  .PHONY: clean
  clean:
  	del *.exe *.o
\end{lstlisting}

\section{配置环境变量的批处理文件}

使用 \texttt{mingw.bat} 文件配置环境变量,其中 \texttt{MinGWPATH} 可以根据实际情况进行修改。

\begin{lstlisting}[language = bash]
  echo off
  set MinGWPATH=D:\Program Files\mingw-w64\x86_64-8.1.0-posix-seh-rt_v6-rev0\mingw64\bin
  set PATH=%MinGWPATH%;%PATH%
  copy "..\makefile"
  copy "..\make.bat"
  call make %1
  IF "%1"=="" GOTO default
  IF "%1"=="clean" GOTO end
  
  %1.exe
  GOTO end
  
  :default
  main.exe
  GOTO end
  
  :end
\end{lstlisting}

\section{负责编译的批处理}

编译文件依旧使用 \texttt{make.bat} 文件。

\begin{lstlisting}[language = bash]
  echo off
  IF "%1"=="" GOTO default
  IF "%1"=="clean" GOTO clean
  
  mingw32-make.exe FILENAME=%1
  GOTO end
  
  :default
  mingw32-make.exe
  GOTO end
  
  :clean
  mingw32-make.exe clean
  :end
\end{lstlisting}

\section{使用方法}

建立一文件夹 \texttt{work\_c}。将以上两个批处理文件和 \texttt{makefile} 文件放入其中。以后每新建一个 C 语言工程,先将 \texttt{mingw.bat} 拷贝至工作目录,然后编写 C 文件,建议文件名与函数名一致。编写完毕,双击 \texttt{mingw.bat} 即可。

\end{document}